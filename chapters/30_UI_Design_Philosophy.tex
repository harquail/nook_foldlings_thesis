\chapter{Design}

\section{Design Philosophy}\label{design-philosophy}

Our design philosophy is simple: follow the user
(\citet{bell2008design}). As often as possible, we presented our
interface concepts to potential users, and allowed their feedback to
guide the design process through the final prototype.

Roughly following the Agile Methodology, we designed and developed the
application collaboratively (\citet{martin2003agile}). Our process was
driven by user experience design, rather than graphic design. That is,
we spent relatively little time polishing aesthetic interface details,
and instead focussed on the core interactions of the application.

Throughout the design process, we explored the conflict between
creativity and rigid geometric constraints. That is, interfaces that
give the user more room for creativity generally tend to make it more
difficult to create designs that will fold correctly in 3D. We aimed for
an interface that provides a great deal of flexibility and creativity,
while guaranteeing that all popup card designs created with it are
valid. Two primary systems maintain this balance.

\begin{enumerate}
\def\labelenumi{\arabic{enumi})}
\itemsep1pt\parskip0pt\parsep0pt
\item
  Tool-based feature creation allows for complex geometry while solving
  geometric constraints transparently.
\item
  The validity system disallows geometry that would interfere with
  existing design elements.
\end{enumerate}

The iPad (and tablets in general) presents unique affordances\footnote{Affordances
  being possible interactions. The term was popularized by
  \citet{norman2013design}.}. We chose to design for the iPad as a way
of forcing the interface to remain minimal and intuitive. With a simple,
gesture-based interface, we could not rely on complex interactions and
instead focussed on distilling the essential design elements. One
benefit of designing for a touch-based interface is that the
interactions are ``natural'' ---~that is, intuitive and easy to learn;
however, gestural interfaces present a unique set of design
problems\footnote{For example, gestural interfaces have relatively few
  inputs: compared to a mouse an keyboard, which have more than 100
  unique input options, there are fewer than 12 gestures recognized by
  Apple's \emph{UIGestureRecognizer} class.}
(\citet{norman2010gestural}). Additionally, the iPad lends itself well
to casual, fun experiences, which aligns with our goal of creating a
usable popup-card design interface for a general audience
(\citet{johansen2013ipad}).

We strive to design an interface that is \textbf{modular},
\textbf{friendly}, and \textbf{delightful}. \textbf{Modularity} stems
from the conception of the pop-up card as a collection of discrete units
that can be acted on individually. Modularity allows users to think in
terms of shape constructions, without concern for individual cuts and
folds. Users can modify, add, and delete individual geometric units,
without affecting the majority of their design. \textbf{Friendliness} is
seen in the careful structuring of our experience to make getting
started as painless as possible. For example, we structure our tutorial
not as a step that must be completed before using the app, but as a
series of brief videos that appear when using a tool for the first time.
\textbf{Delight} comes from small, unexpected details that enhance the
user experience. For example, our color scheme for planes is inspired by
the colors of construction paper. Through this color scheme, we hope to
evoke the spirit of fun and exploration associated with casual
paper-craft.
