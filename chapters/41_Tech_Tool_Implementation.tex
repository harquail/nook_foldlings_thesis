\section{Tool Implementation}\label{tool-implementation}

Below is the definition of the FoldFeature superclass --- all features
created using Foldlings can override these methods to provide specific
functionality. For further discussion of the FoldFeature data structure,
see \textbf{\textgreater{}\textgreater{}TODO cite}.

\singlespacing 

\begin{pygmented}{swift}
var horizontalFolds:[Edge] = [] //list of horizontal folds
var featureEdges:[Edge]?        //edges in a feature
var children:[FoldFeature] = [] // children of feature
var drivingFold:Edge? // driving fold of feature
var parent:FoldFeature? // parent of feature
var startPoint:CGPoint?
var endPoint:CGPoint? // start and end touch points

/// splits an edge around the current feature
func splitFoldByOcclusion(edge:Edge) -> [Edge]
{
//by default, return edge whole
return [edge]
}
/// features are leaves if they don't have children
func isLeaf() -> Bool
{
return children.count == 0
}
/// options or modifications that can be made to the current feature
func tapOptions() -> [FeatureOption]?
{
  return [FeatureOption.PrintPlanes, FeatureOption.PrintEdges,
  FeatureOption.ColorPlaneEdges, FeatureOption.PrintSinglePlane]
}
/// whether a feature is drawn over a fold, determines whether 
/// a fold can be the driving fold for a feature
  func featureSpansFold(fold:Edge)->Bool
{
  return false
}
/// returns and calculates planes in a feature
func getFeaturePlanes()-> [Plane]{
  return featurePlanes
}
/// whether a feature contains a point
/// needs to be overridden by subclasses
func containsPoint(point:CGPoint) -> Bool{
  return self.boundingBox()?.contains(point) ?? false
}
\end{pygmented}

\doublespacing

\subsection{Box Fold}\label{box-fold}

talk about fold heights talk about occlusion

\subsubsection{FreeForm}\label{freeform}

talk about truncation talk about splitting

\subsubsection{Polygon}\label{polygon}

contrast with free-form talk about point dragging talk about truncation

\subsubsection{V-Fold}\label{v-fold}

talk about angle calculation
