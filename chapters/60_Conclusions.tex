\chapter{Conclusions}

We plan to release Foldlings in the Apple app store in September. In
many ways, the primary test of our software will be the extent to which
real users are able to achieve their design goals. However, we can make
several conclusions about the success of our tool. In general, users
find the process of designing popup cards more intuitive with the
current version of Foldling than with previous versions of our software
or manual methods. Additionally, we have qualitative evidence that
suggest that users can create a wide range of complex cards, faster and
with more precision than using manual methods. That said, there is much
work to be done on this and related popup-card design problems.

\section{User Interface Future Work}\label{user-interface-future-work}

\subsection{Modifications to the Master
Card}\label{modifications-to-the-master-card}

Currently, our software only allows for one size and type of master card
feature. That is, a greeting-card sized piece of paper, with a single
driving fold in the center. Allowing modifications to the driving fold
might allow users to change paper size, rotate the card so that the
middle fold is vertical, or construct a diagonal fold for the master
card.

\subsection{Multiple Cards}\label{multiple-cards}

Currently, our software. In order to support combinations of
interlocking sketches, we would need to create an interface that allow
for connecting pop-up card elements in 3D. Although very complex to
implement, this interface could ultimately allow for a far more complex
arrangement of features that our software currently affords
(\citet{hart2007modular}).

\subsection{Safe Area Guides}\label{safe-area-guides}

Often, users wish construct a fully-contained popup card. That is, a
card that can close fully, with no portions of internal fold features
visible when the card is closed. In order to achieve this design in a
symmetrical card, all features with a driving fold must be within the
middle third of the card. We could add ``safe area'' guides and warnings
to indicate this area to users that want add that additional constraint
to their design.

\section{Algorithms \& Implementation Future
Work}\label{algorithms-implementation-future-work}

\subsection{Feature Intersections}\label{feature-intersections}

Feature intersections are only partially implemented, and do not always
succeed. To fully-implement feature intersections, we would need to
refactor our FoldFeature class to add feature intersections a primary
component. This would replace the current method of intersecting
features with folds --- \emph{splitFoldByOcclusion} --- and would allow
for more generalizable intersections between features.

\subsection{Concurrency}\label{concurrency}

A key limitation of Foldlings is that all functions currently run on a
single thread. As a consequence, the user is sometimes blocked by
operations that could be performed in the background. For example, when
completing a feature, out app ignores touch input until the feature is
added to the sketch and planes are calculated. This can cause a slight
but noticeable delay between Restructuring our algorithms to perform
computationally-heavy operations in the background would reduce lag
between action, allowing users to design more quickly and fluidly.
