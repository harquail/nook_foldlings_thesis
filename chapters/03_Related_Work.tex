\section{Related Work}\label{related-work}

\textbf{\textgreater{}\textgreater{}TODO: Complete w/Marissa}

\emph{This section is co-authored with Marissa Allen}

Our approach to pop-up creation does not require glue or 3D modeling
experience, but is more similar to kirigami. Glue-based design presents
different affordances than kirigami, but requires an assembly stage that
disconnects the initial 2D pattern from the final pop-up geometry
{[}Glassner 1998{]}. Others approach the pop-up design problem from the
opposite direction: creating designs by modeling in 3D space {[}Ruiz et
al. 2014{]}. {[}Li et al. 2010{]} transform 3D models into paper
architectures that are stable and rigid. A drawback to this approach is
that it does not preserve the original model's features. Compared to
existing tools, Foldlings allows novice users more opportunities for
free-form, artistic expression in the pop-up medium.

There has been prior work done on automated methods to create a pop-up
schematic. For example, Li, Xian-Ying, et al. develop an algorithm that
transforms user-defined models into paper architectures that are stable
and rigid. Because their algorithm modifies the input geometry given by
the user, the end result does not preserve the original model's
features. Other attempts to create pop-up cards such as Abel, Zachary,
et al. --- can only take in simple polygonal meshes and require the user
to fold and glue additional pieces of paper together, whereas our
approach creates a pop-up card with arbitrary shapes from a single piece
of paper. These methods also impose strict constraints on user input,
which requires the end user to know how to model an object in 3D and
thus is not accessible to novice users.

Our approach to smoothing user input as the user draws his/her design is
similar to Johnson, Gabe, et al. After a user has drawn lines in their
program, they edit these lines depending on what mode the user is in and
the drawing is then sent to a laser-cutter. However, our method
automatically interpolates the lines given by the user and transforms
them into smoothed Bezier curves, ready to be printed.

Hendrix, Susan L., and Michael Eisenberg. ``Computer-assisted pop-up
design for children: computationally enriched paper engineering.''
Advanced Technology for Learning 3.2 (2006): 119-127.
\citet{hendrix2006computer}

The pop-up tool described by this paper is very similar to our current
design. However, this tool was built as an introduction to the
engineering sciences and not as a tool to aid artistic design. Perhaps
as a result of an engineering-centric design methodology, the pop-up
tool has a more restrictive UI than ours. Their program is designed to
only create the simple folding mechanism of the pop-up card. Adding
arbitrary cuts and multiple layers is not supported and must be added
later.

Okamura, Sosuke, and Takeo Igarashi. ``An interface for assisting the
design and production of pop-up card.'' Smart Graphics. Springer Berlin
Heidelberg, 2009. \citet{okamura2009interface}

This paper describes a program to design elaborate 180º pop-ups that can
be applied to both cards and books strictly within a 3D environment.
Their manufacturing process is additive, using glue to connect the
pieces. They implement collision testing and multiple components inside
their pop-up program. However, their program does not incorporate the
main folding piece in designing their pop-ups, it is simply a mechanical
driver for the rest of the design. Therefore, a user cannot design a
pop-up that consists of only one piece of paper that does not require
gluing.

Way, Der-Lor, Yong-Ning Hu, and Zen-Chung Shih. ``The Creation of V-fold
Animal Pop-Up Cards from 3D Models Using a Directed Acyclic
Graph.''Advances in Intelligent Systems and Applications-Volume 2.
Springer Berlin Heidelberg, 2013. 465-475. \citet{way2013creation}

The tool described in this paper creates pop-ups from 3D models. The
tool segments a 3D model and then uses shape recognition to create paper
models in pop-up cards. The authors create an acyclic graph from these
segments and use the nodes to drive a simulation of the pieces based on
opening and closing the card. This paper has a similar simulation
technique to our approach. However, they generate their designs from 3D
shapes instead of 2D patterns.
