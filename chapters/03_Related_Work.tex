\section{Related Work}\label{related-work}

\emph{This section is co-authored with Marissa Allen}

Software that creates 3D geometry from 2D sketches is an active and
vibrant area of HCI and graphics research. For example,
\citet{igarashi2007teddy} present a sketch-based interface that creates
3D geometry in real time based on 2D silhouettes sketched by users.
\citet{patrick20022d} use a silhouette-based approach in a tool for
searching 3D models, and \citet{wang2003feature} use 2D sketching to
design 3D garments. We solve a far more tightly-constrained problem, in
that we are concerned with foldable pop-up kirigami, rather than
arbitrary 3D meshes.

Specifically, there has been some previous work in pop-up card design
software. The seminal work on popup-card design software,
\citet{glassner1998interactive}, uses glue to connect components, and so
does not meet our criteria for a kirigami pop-up card. Still, we take
inspiration from Glassner's work, and borrow his calculation for
resolving the v-fold angle constraint. Our approach to pop-up creation
does not require glue or 3D modeling experience, but is strict kirigami,
as described in \nameref{background} on page \pageref{background}.
Glue-based design presents different affordances than kirigami, but
requires an assembly stage that disconnects the initial 2D pattern from
the final pop-up geometry.

Others approach the pop-up design problem from the opposite direction:
creating designs by modeling in 3D space \citet{ruiz2014multi}. For
example, \citet{li2010popup} develop an algorithm that transforms
user-defined models into paper architectures that are stable and rigid.
Because their algorithm modifies the input geometry given by the user,
the end result does not necessarily preserve the user's design intent.
Other attempts to create pop-up cards such as \citet{abel2013algorithms}
can only take in simple polygonal meshes and require the user to fold
and glue additional pieces of paper together, whereas our approach
creates a pop-up card with arbitrary shapes from a single piece of
paper. These methods all require the end user to know how to use
traditional 3D-modeling software, and thus are not accessible to novice
users. Another drawback of these is relying on 3D model as input allows
for less exploration on the part of the user. Because constructing
designs in 3D requires significantly more effort than creating a 2D
sketch, users are less likely to explore many design possibilities, and
therefore less likely to gain an intuitive understanding of the
geometric constraints imposed by paper.

\citet{way2013creation} also create pop-ups from 3D models. The tool
segments a 3D model and then uses shape recognition to create paper
models in pop-up cards. However, their method of simulation is similar
enough to our's to merit special mention. The authors create an acyclic
graph from paper segments and use the nodes to drive a simulation of the
pieces based on opening and closing the card. We also create an acyclic
graph of nodes, and rotate planes during simulation based on traversal
of the graph.

Others approach a slightly-different pop-up card design problem.
\citet{li2011geometric} present algorithms for creating v-style pop-ups.
Not to be confused with v-folds, they create pop-ups composed of
``patches falling into four parallel groups,'' and present algorithms
for constructing and analyzing v-style popup books. V-style planes have
different angle constraints than our orthogonal pop-ups, allowing for a
completely different set of designs fulfilling geometric constraints.
Their designs require multiple sheets of paper --- attached via glue or
with a paper hinge mechanism. \citet{okamura2009interface} describe a
program to design elaborate 180º pop-ups that can be applied to both
cards and books strictly within a 3D environment. They implement
collision testing between multiple components inside their pop-up
program. However, their program does not incorporate the card's primary
fold in designing pop-ups, it is simply a mechanical driver for the rest
of the design, which is attached using glue. Therefore, their designs
are not kirigami.

\citet{xu2007computer} describe algorithms for creating 2D cut patterns
from input images, allowing designs to be composed from multiple
composited shapes. Although their result is strictly two-dimensional, we
share their interest in producing paper art through software. Similarly,
\citet{johnson2012sketch} present an interface for creating precise
laser-cut designs by interpreting and smoothing user sketches. Although
not directly related to the pop-up card design problem, we use a similar
method for smoothing user input.

In addition to pop-up card design, several authors create tools for
origami design. That is, designs created using only folds.
\citet{fastag2009egami} describe ``eGami,'' an interactive system for
folding flat (zero-thickness) origami models. While they solve a very
different problem than Foldlings, they take a similar approach in terms
of providing a toolset of common operations and displaying an
interactive preview. \citet{kasem2008computational} present a web-based
tool for origami design, built on the work of
\citet{zamiatina1994computer}. \citet{ju2002origami} take an unusual
approach to origami design software, in that the user physically folds
the card, while projectors and a sensing system provide feedback on how
closely the user's actions match a target design. While this kind of
feedback is useful in reproducing an existing design, it does not
facilitate the creation of novel designs.

\citet{hendrix2006computer} describe a pop-up design interface with
goals very similar to those of Foldlings --- they aim to help users
create valid designs and visualize their constructions in 3D. However,
this tool was built as an introduction to the engineering sciences and
not as a tool to aid artistic design. Perhaps as a result of an
engineering-centric design methodology, the pop-up tool has a more
restrictive UI than ours. Their program is designed to only create the
simple folding mechanism of the pop-up card. Non-rectilinear cuts must
be added manually, outside the software. In comparison, our tool allows
for more creativity, by allowing the user far more control over the
design and the ability to preview the entirety of the design, not just
the folding structure. Additionally, while their software includes
feature-based tools, their approach does not allow for the modular
modification of design elements after creation.
