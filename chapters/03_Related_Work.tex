\section{Related Work}\label{related-work}

\emph{This section is co-authored with Marissa Allen}

Software that creates 3D geometry from 2D sketches is an active and
vibrant area of HCI and graphics research. For example,
\citet{igarashi2007teddy} present a sketch-based interface that creates
3D geometry in real time based on 2D silhouettes sketched by users.
\citet{patrick20022d} use a similar silhouette-based approach in a tool
for searching 3D models, and \citet{wang2003feature} use 2D sketching to
design 3D garments. We solve a far more tightly-constrained problem, in
that we are concerned with foldable pop-up kirigami, rather than
arbitrary 3D meshes.

Specifically, there has been some previous work in pop-up card design
software.

Our approach to pop-up creation does not require glue or 3D modeling
experience, but is strict kirigami, as described in \nameref{background}
on page \pageref{background}. Glue-based design presents different
affordances than kirigami, but requires an assembly stage that
disconnects the initial 2D pattern from the final pop-up geometry
(\citet{glassner1998interactive}). Still, we take inspiration from
Glassner's work, and borrow his calculation for resolving the v-fold
angle constraint.

ADD Li et al 2011

\textbf{\textgreater{}\textgreater{}3D to 2D:} Others approach the
pop-up design problem from the opposite direction: creating designs by
modeling in 3D space {[}Ruiz et al. 2014{]}. {[}Li et al. 2010{]}
transform 3D models into paper architectures that are stable and rigid.
A drawback to this approach is that it does not preserve the original
model's features. Compared to existing tools, Foldlings allows novice
users more opportunities for free-form, artistic expression in the
pop-up medium.

egami. Origami Design. \citet{fastag2009egami}

\citet{xu2007computer} describe algorithms for creating 2D cut patterns
from input images, allowing designs to be composed from multiple
composited shapes. Although their result is strictly two-dimensional, we
share their interest in producing paper art through software. Similarly,
\citet{johnson2012sketch} present an interface for creating precise
laser-cut designs by interpreting user sketches. Although not directly
related to the pop-up card design problem, we use a similar method for
smoothing user input.

\textbf{\textgreater{}\textgreater{}3D to 2D:} There has been prior work
done on automated methods to create a pop-up schematic. For example, Li,
Xian-Ying, et al. develop an algorithm that transforms user-defined
models into paper architectures that are stable and rigid. Because their
algorithm modifies the input geometry given by the user, the end result
does not preserve the original model's features. Other attempts to
create pop-up cards such as Abel, Zachary, et al. --- can only take in
simple polygonal meshes and require the user to fold and glue additional
pieces of paper together, whereas our approach creates a pop-up card
with arbitrary shapes from a single piece of paper. These methods also
impose strict constraints on user input, which requires the end user to
know how to model an object in 3D and thus is not accessible to novice
users.

\citet{hendrix2006computer} describe a pop-up tool very similar to
Foldlings. However, this tool was built as an introduction to the
engineering sciences and not as a tool to aid artistic design. Perhaps
as a result of an engineering-centric design methodology, the pop-up
tool has a more restrictive UI than ours. Their program is designed to
only create the simple folding mechanism of the pop-up card.
Non-rectilinear cuts must be added manually, outside the software.

Okamura, Sosuke, and Takeo Igarashi. ``An interface for assisting the
design and production of pop-up card.'' Smart Graphics. Springer Berlin
Heidelberg, 2009. \citet{okamura2009interface}

This paper describes a program to design elaborate 180º pop-ups that can
be applied to both cards and books strictly within a 3D environment.
Their manufacturing process uses glue to connect the pieces. They
implement collision testing and multiple components inside their pop-up
program. However, their program does not incorporate the card's primary
fold in designing pop-ups, it is simply a mechanical driver for the rest
of the design. Therefore, a user cannot design a pop-up that consists of
only one piece of paper that does not require gluing.

Way, Der-Lor, Yong-Ning Hu, and Zen-Chung Shih. ``The Creation of V-fold
Animal Pop-Up Cards from 3D Models Using a Directed Acyclic
Graph.''Advances in Intelligent Systems and Applications-Volume 2.
Springer Berlin Heidelberg, 2013. 465-475. \citet{way2013creation}

\textbf{\textgreater{}\textgreater{}3D to 2D:} The tool described in
this paper creates pop-ups from 3D models. The tool segments a 3D model
and then uses shape recognition to create paper models in pop-up cards.
The authors create an acyclic graph from these segments and use the
nodes to drive a simulation of the pieces based on opening and closing
the card. This paper has a similar simulation technique to our approach.
However, they generate their designs from 3D shapes instead of 2D
patterns.

Another approach is to use physical.
