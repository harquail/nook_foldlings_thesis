\section{Interface Data Structures}\label{interface-data-structures}

\subsection{Edges}\label{edges}

An Edge represents a cut or fold. Edges are the basic building block of
planes and

\subsubsection{Driving Folds}\label{driving-folds}

A driving fold is not a special type of edge, but rather a . Any fold
can be the driving fold for

\subsection{Planes}\label{planes}

A plane is a list of Edges. \textbf{TODO: CITE MARISSA HERE}

\subsection{Fold Features}\label{fold-features}

The central data structure of Foldlings is the FoldFeature: a
representation of a shape drawn by the user that folds in 3d. Each fold
feature is a single design element ---~and can be individually created,
modified, and deleted. There are five types of FoldFeature: MasterCard,
BoxFold, FreeForm, Polygon, and V-Fold, representing differences in
drawing behavior, geometry, and (the differences are described in detail
below). Each of these features is a subclass of the FoldFeature super
class.

All FoldFeatures have functionality in common:

\begin{itemize}
\itemsep1pt\parskip0pt\parsep0pt
\item
  Each feature contains a list of edges in the feature ---~both cuts and
  folds
\item
  Each feature has a driving fold --- in the case of unconnected
  features, such as the master card and holes, the driving fold is nil.
\item
  Each feature can be deleted from the Sketch, ``healing'' the sketch by
  closing gaps left in any
\item
  Features implement the encodeWithCoder and decodeWithCoder methods,
  allowing them to be serialized to a file on the device and restored
  from the saved file.
\item
  Each feature can provide a list of current ``tap options'' --- actions
  that can be performed on the feature given its state. \textbf{TODO:SEE
  tap options in interface design} \citep{Nobody06}. \textbf{TODO:REMOVE
  -- JUST TESTING}
\end{itemize}

In addition, each feature contains

\subsubsection{MasterCard}\label{mastercard}

\subsubsection{Box Fold}\label{box-fold}

\subsubsection{FreeForm}\label{freeform}

Holes are a special case of FreeForm shapes. FreeForm shapes that do not
cross a fold are considered holes ---~drawn in white in the 2d sketch
and drawn as subtractions from planes in the 3d view.

\subsubsection{Polygon}\label{polygon}

\subsubsection{V-Fold}\label{v-fold}

\subsubsection{Validity}\label{validity}
