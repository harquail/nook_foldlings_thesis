\section{Potential Applications}\label{potential-applications}

\emph{This section is co-authored with Marissa Allen}
\textbf{\textgreater{}\textgreater{}TODO: get comments/more content from
marissa}

As a general-purpose design tool for cuts and folds, Foldlings has a
wide variety of potential applications. For example, Melina Blees et al
present a graphene transistor that is constructed through kirigami
methods (\citet{blees2014graphene}). Simple, usable interfaces for
designing complex kirigami structures are needed to advance similar
research.

One exciting application of Foldlings is as a tool for developing
advanced spatial reasoning skills. Taylor et al present a curriculum
that uses popup card design as a tool for building mathematics and
spatial reasoning skills \citet{taylor2013think3d}
(\citet{olson_mathematics_2004}). A tool like Foldlings would likely
increase the effectiveness of such a program, since much of the
experimentation could happen more quickly in software than using manual
methods. Because our code is open source, advanced students could even
modify our software to develop new fold features and interactions.
